% Created 2018-04-14 Sat 21:50
% Intended LaTeX compiler: pdflatex
\documentclass[11pt]{article}
\usepackage[utf8]{inputenc}
\usepackage[T1]{fontenc}
\usepackage{graphicx}
\usepackage{grffile}
\usepackage{longtable}
\usepackage{wrapfig}
\usepackage{rotating}
\usepackage[normalem]{ulem}
\usepackage{amsmath}
\usepackage{textcomp}
\usepackage{amssymb}
\usepackage{capt-of}
\usepackage{hyperref}
\author{Shubham Mahajan}
\date{\today}
\title{Home}
\hypersetup{
 pdfauthor={Shubham Mahajan},
 pdftitle={Home},
 pdfkeywords={},
 pdfsubject={},
 pdfcreator={Emacs 25.2.2 (Org mode 9.1.9)}, 
 pdflang={English}}
\begin{document}

\maketitle
\tableofcontents

\begin{center}
\begin{tabular}{ccc}
\hline
\hline
\href{https://imahajanshubham.github.io/docs/home.html}{Home} & \href{https://github.com/imahajanshubham}{GitHub} & \href{https://sites.google.com/view/the-launchpad/home}{Website}\\
\hline
\end{tabular}
\end{center}

\section{About}
\label{sec:org066cdcd}
Sharing my views and experience of \textbf{Problem Solving} approach via the blog.

\section{{\bfseries\sffamily TODO} Trending}
\label{sec:org5abbe2a}
Coming soon, stay tuned

\section{Latest}
\label{sec:orgeb0483a}
\begin{attention}
Another wonderful opportunity to learn,\\
Yapp, it’s coming!

\textbf{\href{https://imahajanshubham.github.io/docs/others/words\_to\_cherish.html}{The greatest enemy of learning is knowing}}
\end{attention}

\section{C-C++}
\label{sec:orgafd0cb9}
\subsection{\href{https://imahajanshubham.github.io/docs/lang/c-c++/beware\_of\_the\_assumptions.html}{Beware Of The Assumptions}}
\label{sec:org4244125}
\begin{quote}
Calculating the factorials of 100+ using \textbf{Divide and Conquer} approach!
\end{quote}

\textbf{March 8,} 2018

\subsection{\href{https://imahajanshubham.github.io/docs/lang/c-c++/occurrences\_of\_a\_digit.html}{Occurrence of a Digit}}
\label{sec:orgb6ec1f7}
\begin{quote}
The goal was to find the no. of times, \textbf{a particular digit appeared} (occurrences)
in a damn large positive number*.
\end{quote}

\textbf{February 24,} 2018

\subsection{\href{https://imahajanshubham.github.io/docs/lang/c-c++/acid\_naming.html}{Acid Naming}}
\label{sec:org21f00c5}
\begin{quote}
The goal was to identify the type of acids from the given set of inputs.
\end{quote}

\textbf{February 17,} 2018

\subsection{\href{https://imahajanshubham.github.io/docs/lang/c-c++/structures.html}{Structures and Pointers}}
\label{sec:orgabfda19}
\begin{quote}
An attempt to share my thoughts on \textbf{Structures and Pointer} and finally a small
program using them too.
\end{quote}

\textbf{February 7,} 2018

\subsection{Only Source Code}
\label{sec:orgd4dec8f}
\subsubsection{\href{https://imahajanshubham.github.io/docs/lang/c-c++/only\_source\_code/average\_halfify.html}{Average Halfify (keteki)}}
\label{sec:org5b93f9d}
\begin{quote}
The goal was to calculate the \textbf{recursive average of pairs of n numbers}.
\end{quote}

\textbf{March 7,} 2018

\subsubsection{\href{https://imahajanshubham.github.io/docs/lang/c-c++/only\_source\_code/little\_chefnsums.html}{Little Chef and Sums}}
\label{sec:org2a252ac}
\begin{quote}
The goal was calculate the total of \textbf{suffix-sum} and \textbf{prefix-sum} of given \textbf{n} values
in an array.
\end{quote}

\textbf{March 7,} 2018

\subsubsection{\href{https://imahajanshubham.github.io/docs/lang/c-c++/only\_source\_code/magical\_function.html}{Magical Function}}
\label{sec:org5c8c1de}
\begin{quote}
The goal was to \textbf{recognize the pattern} of a given function.
\end{quote}

\textbf{March 7,} 2018

\subsubsection{\href{https://imahajanshubham.github.io/docs/lang/c-c++/only\_source\_code/minimizing\_the\_dotproduct.html}{Minimizing the Dot Product}}
\label{sec:org356758f}
\begin{quote}
The goal was to calculate the \textbf{minimum dot product of given two vectors}. we could
interchange the vector positions with each other if needed. 
\end{quote}

\textbf{March 7,} 2018

\section{Java}
\label{sec:org7fd077a}
\subsection{\href{https://imahajanshubham.github.io/docs/lang/java/factorial.html}{Factorials in Java}}
\label{sec:org4e91cf9}
\begin{quote}
The goal was to calculate the factorial of say… \texttt{50}, which results in a \texttt{65}
digit answer.
\end{quote}

\textbf{February 18,} 2018

\subsection{\href{https://imahajanshubham.github.io/docs/lang/java/hashmap.html}{Hash Maps in Java}}
\label{sec:org15744ce}
\begin{quote}
a brief introduction about hash maps in java, starting from basic definition,
properties, syntax to creating a simple program of phonebook in java.
\end{quote}

\textbf{February 9,} 2018

\section{Others}
\label{sec:orgd09769e}
\subsection{\href{https://imahajanshubham.github.io/docs/others/words\_to\_cherish.html}{The greatest enemy of learning is knowing}}
\label{sec:orga8de196}
\begin{quote}
"The man who asks a question is a fool for a minute, the man who does not ask is a fool for life."

― Confucius
\end{quote}

\textbf{March 30,} 2018

\section{Thank you}
\label{sec:org9270dd0}
\noindent\rule{\textwidth}{0.5pt}

\begin{attention}
If this blog was worth your time, please do checkout my other \href{https://imahajanshubham.github.io/docs/home.html}{shares} too.
\end{attention}

Wanna know more about me?

Visit:

\begin{center}
\begin{tabular}{ll}
\hline
\textbf{LinkedIn} & \href{https://www.linkedin.com/in/imahajanshubham/}{imahajanshubham}\\
\textbf{Website} & \href{https://sites.google.com/view/the-launchpad/home}{The Launchpad}\\
\hline
\end{tabular}
\end{center}

or where my work lives:

\begin{center}
\begin{tabular}{ll}
\hline
\textbf{GitHub} & \href{https://github.com/imahajanshubham}{imahajanshubham}\\
\hline
\end{tabular}
\end{center}

Want to share something with me or just wanna chat?

\begin{center}
\begin{tabular}{ll}
\hline
\textbf{Mail me} & \href{mailto:imahajanshubham@gmail.com?}{imahajanshubham@gmail.com}\\
\hline
\end{tabular}
\end{center}

\noindent\rule{\textwidth}{0.5pt}
\end{document}
